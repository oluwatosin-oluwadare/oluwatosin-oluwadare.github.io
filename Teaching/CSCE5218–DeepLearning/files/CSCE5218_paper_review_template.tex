% Minimal CVPR-style template for paper reviews
% Based on CVPR 2026 author kit (simplified for CSCE 5218)

\documentclass[10pt,letterpaper]{article}

% CVPR style (camera-ready style, not review mode)
\usepackage{cvpr}

\definecolor{cvprblue}{rgb}{0.21,0.49,0.74}
\usepackage[pagebackref=false,breaklinks=true,colorlinks=true,allcolors=cvprblue]{hyperref}

% Optional basic math / symbols
\usepackage{amsmath,amssymb}

% Force one-column layout (simplifies reading and grading)
\onecolumn

%%%%%%%%% COURSE INFO (instructor can edit if needed) %%%%%%%%%
\def\confName{CSCE 5218 -- Deep Learning}
\def\confYear{2025}

%%%%%%%%% PAPER INFO (students must edit) %%%%%%%%%
\title{Paper Review: ``[Paper Title Here]''}

\author{
  FirstName LastName\\
  CSCE 5218 -- Deep Learning\\
  University of North Texas\\
  {\tt\small student@my.unt.edu}
}

\begin{document}
\maketitle

%%%%%%%%%%%%%%%%%%%%%%%%%%%%%%%%%%%%%%%%%%%%%%%%%%%%%%%%%
\section{Summary}
Describe the paper’s goal, method, and results \textbf{in your own words}.

\noindent\textbf{Guiding questions:}
\begin{itemize}
    \item What problem does the paper address?
    \item What method or architecture does it introduce?
    \item What were the major results?
\end{itemize}

% (Students replace this paragraph with their own summary.)
Write your summary here. This section should briefly explain the main idea of the paper,
the approach used, and the key findings.

%%%%%%%%%%%%%%%%%%%%%%%%%%%%%%%%%%%%%%%%%%%%%%%%%%%%%%%%%
\section{Three Key Things You Learned}
List and explain at least three important concepts, techniques, or lessons you gained from the paper.

\noindent\textbf{Examples:}
\begin{itemize}
    \item ``I learned how convolution filters extract features from images and build hierarchical representations.''
    \item ``I learned that residual connections help with gradient flow in very deep networks.''
    \item ``I learned why pretraining on large datasets improves downstream performance.''
\end{itemize}

% (Students should write at least three bullet points with short explanations.)
Write your three key takeaways here. Explain why each is important or interesting.

%%%%%%%%%%%%%%%%%%%%%%%%%%%%%%%%%%%%%%%%%%%%%%%%%%%%%%%%%
\section{New Knowledge}
Identify ideas, terms, or methods that were \textbf{new to you} and describe how they expanded your understanding.

\noindent\textbf{Guiding questions:}
\begin{itemize}
    \item What concepts or techniques were unfamiliar before reading?
    \item What new tools, datasets, or architectures did you discover?
    \item What results or analysis surprised you?
\end{itemize}

Write this section focusing on what you did \emph{not} know before reading the paper and how it helped you learn.

%%%%%%%%%%%%%%%%%%%%%%%%%%%%%%%%%%%%%%%%%%%%%%%%%%%%%%%%%
\section{Questions or Areas for Improvement}
Discuss parts of the paper that were unclear, confusing, or that you think could have been explained better from a student perspective.

\noindent\textbf{Examples:}
\begin{itemize}
    \item ``I found the mathematical notation unclear in Section 3.''
    \item ``The dataset description was too brief; I wanted more details about preprocessing.''
    \item ``I didn’t understand why they chose this particular baseline model.''
\end{itemize}

Write your questions and constructive suggestions here. Be specific rather than giving generic comments.

%%%%%%%%%%%%%%%%%%%%%%%%%%%%%%%%%%%%%%%%%%%%%%%%%%%%%%%%%
% OPTIONAL: REFERENCES (if students cite other work)
%%%%%%%%%%%%%%%%%%%%%%%%%%%%%%%%%%%%%%%%%%%%%%%%%%%%%%%%%

% If you need references beyond the main paper, you may uncomment this:
%{
%    \small
%    \bibliographystyle{ieeenat_fullname}
%    \bibliography{main}
%}

\end{document}
